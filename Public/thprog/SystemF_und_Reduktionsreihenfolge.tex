	\documentclass{article}
	\usepackage{amsmath}
	\usepackage{nccmath}
	\DeclareMathSizes{10}{10}{10}{10}
	\setlength{\parindent}{0pt}
	\title{System F und Reduktionsreihenfolge}
	\date{ }
	\begin{document}
		\section{Generelles}
		\subsection{Normale-/Lazy-Reduktion}
			- pre-order durch Baum ("von unten nach oben auswerten")	\\
			- leftmost-outermost\\
			- Argumente zum Schluss auswerten
			\begin{align*}			
				pow(a,pow(c,b))\:\:	& mit\;linkem\;pow\;anfangen\\
									& dann\;a\\
									& dann\;rechtes\;pow\\
									& dann\;c\\
									& dann\;d
			\end{align*}
		\subsection{Applikative Reduktion:}
			- post-order durch Baum ("von oben nach unten")\\
			- leftmost-innermost\\
			- als erstes die Argumente auswerten
			\begin{align*}
				pow(a,pow(c,b))\:\: & mit\;c\;anfangen
				\enspace\;\;\;\;\;\;\;\;\;\enspace\;\;\;\;
											\\
									& dann\;b\\
									& dann\;a\\
									& dann\;rechtes\;pow\\
									& dann\;linkes\;pow
			\end{align*}
		\section{Typherleitung}
			\subsection{Regeln}
				
		
		
		
		
		
		
		
		
		
		
		
		
		
		
		
		
		
		
		
		
		
	\begin{tiny}
	\copyright\ Joint-Troll-Expert-Group (JTEG) 2015
	\end{tiny}
\end{document}