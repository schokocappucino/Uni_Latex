	\documentclass{article}
	\usepackage{amsmath}
	\usepackage{nccmath}
	\DeclareMathSizes{10}{10}{10}{10}
	\setlength{\parindent}{0pt}
	\title{Ko-Rekursion/-Induktion}
	\date{ }
	\begin{document}
	%\maketitle
	\section{Korekursion}
	\textbf{Gegeben:}
		\begin{fleqn}
				\begin{align*}
					co&data Signal\:where  \\
					& \enspace currentSample : Signal \rightarrow Int \\
					& \enspace discardSample : Signal \rightarrow Signal
				\end{align*}
		\end{fleqn}
	\textbf{sowie:}
	\begin{fleqn}
	\begin{align*}
		&currentSample(flat\:x) = x			&
		&discardSample(flat\:x) = flat\:x	\\
		&currentSample(square\:x\:y) = x		&
		&discardSample(square\:x\:y) = square\;y\;x	
	\end{align*}	
	\end{fleqn}
	\textbf{Es soll gelten:}\\
	sampler t s\enspace = s , wenn $t>0$ und 0 sonst\\
	sowie insbesondere:
	\begin{fleqn}
	\begin{align*}
		&sampler:\:Signal\rightarrow Signal\rightarrow Signal \\
		&sampler(square\:0\:1)(square\:x\:0) = flat\:0 \\
		&sampler(square\:1\:0)(flat\:x) = square\:x\:0  \\
	\end{align*}	
	\end{fleqn}
	\textbf{Schritt 1:}\\
	$\rightarrow$ sampler Funktion in codata-Funktionen einsetzen
	\begin{fleqn}
	\begin{align*}
		&currentSample(sampler\:t\:s)\\
		&discardSample(sampler\:t\:s)	
	\end{align*}	
	\end{fleqn}
	\textbf{Schritt 2:}\\
	$\rightarrow$ Bedingung von oben bei erster Funktion anwenden also:
	\begin{fleqn}
		\begin{align*}
			currentSample(sampler\:t\:s) = \:&if\:(currentSample\:\:t>0)\\
											&then\:(currentSample\:\:s)\\
											&else\:0
		\end{align*}
	\end{fleqn}
	\textbf{sowie zweite Formel zum Aufteilen verwenden:}
		\begin{fleqn}
		\begin{align*}
			discardSample(sampler\:t\:s) = sampler(discardSample\:\:t)(discardSample\:\:s)
		\end{align*}
	\end{fleqn}
	\newpage
	\section{Ko-Induktion:}
	\textbf{Induktionsanfang:}\\
	$\rightarrow$ anhand erster Formel 'R' aufstellen
	\[ 
		R = \{ 
			\underbrace{
				sampler(square\:0\:1)(square\:x\:0)}_{linke\:Seite},\underbrace{flat\:0}_{rechte\:Seite} \:|\: \underbrace{x\in Int}_{von\:oben}\}
		\]
	$\rightarrow$ "R ist Bisimulation" hinschreiben, linke Seite aufloesen\\
	\[ currentSample(\,sampler(square\:0\:1)(square\:x\:0)) = 0 \]
	$\rightarrow$ wenn hier am Ende kein Term rauskommt, dann koennen wir einfach die \textbf{rechte Seite bei der normalen Funktion} einsetzen und selbige aufloesen:
	\[ currentSample(flat\;0) = 0 \]
	$\rightarrow$ sehr gut, die rechten Seiten sind gleich wir sind hier also fertig\\\\
	\textbf{\underline{WICHTIG} Jetzt das selbe noch mit discardSample()!}\\\\
	\textbf{zweite Bedingung:}
	\begin{align*}	
	discardSample(sampler&(square\;1\;0)(square\;0\;x)) \\
	=& sampler(discardSample (square\;1\;0))(discardSample(square\;0\;x))
	\end{align*}
	\textit{das ist doof denn:}
		\[discardSample(square\;x\;0) = discardSample(square\;0\;x)\]
	\textbf{Schritt 3:}\\
	\[R' = R \;\cup\;\{\underbrace{
	sampler(discardSample (square\;1\;0))(discardSample(square\;0\;x))	
	}_{aufgeloeste\;2.\;Bedingung},discardSample(square\;x\;0\,)\:|\:x\in Int\}\]	
	wir muessen jetzt zeigen:
	\[
		\underbrace{currentSample(sampler(square\:1\:0)(square\:0\:x)}_{currentSample(\,x\;0\,)}) = \underbrace{currentSample(square\;0\,x)}_{selbes\;wie\;rechts}
	\]
	passt also, jetzt wie oben auch zweite Funktion:
	\[
		discardSample(sampler(square\:1\:0)(square\:0\:x)) = \underbrace{sampler(square\:1\:0)(square\:0\:x)}_{wieder\;linke\;Seite\;in\;R'}\]\[
		discardSample(\,square\;x\;0\,) = \underbrace{discardSample(\,square\;0\;x\,)}_{wieder\;rechte\;Seite}
	\]
	und fertig!\\\\
\section*{Sonstiges:}
	Eine Relation $\mathrm{R}\;\subset\;A^{\omega}\;$x$\;A^{\omega}$ heisst Bisimulation, wenn f\"ur
	alle $(s,s')$ $\in\; \mathrm{R}$ gilt: \\
	\noindent\hspace{1cm} currentSample s = currentSample s' \;\;\;\; weil Int zur\"uck\\
	\noindent\hspace{1cm} (discardSample\;s)\;R\;(discardSample\;t)\;\;\;\; weil wieder Signal\\\\\\\\

	Wenn R eine Bisimulation ist, dann gilt $\mathrm{R}\;\subseteq\;id$ , d.h.
	\[
		sRt \Rightarrow s = t
	\]
	f\"ur alle s,t $\in\;A^\omega$\\\\
	\begin{tiny}
	\copyright\ Joint-Troll-Expert-Group (JTEG) 2015
	\end{tiny}
	\newpage
	
	
	
	
	
	
	
	
	
	
	
	
	
	
	
	
\end{document}
