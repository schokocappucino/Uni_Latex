	\documentclass{article}
	\usepackage{amsmath}
	\usepackage{nccmath}
	\DeclareMathSizes{10}{10}{10}{10}
	\setlength{\parindent}{0pt}
	\title{Konfluenz}
	\date{ }
	\begin{document}
	\section*{Wichtige Lemmas}
		\subsection*{Newman's Lemma:}
			Ein stark normalisierendes und lokal Konfluentes Termersetzungssystem (TES) ist konfluent.
		\subsection*{}
		\section{Matching Table:}
			Formeln:
			\begin{align}
				x \Uparrow ( y \Uparrow z) & \rightarrow_{0}
					\; x \Uparrow (y \Downarrow y)
				\\				
				x \Downarrow ( x \Downarrow y ) & \rightarrow_0
					\; x \Downarrow y
			\end{align}
			- alle Regeln m\"ussen gegen alle anderen gematched werden 
			daher f\"ur \"ubersichtlichkeit:\\ \\
			\begin{tabular}{l c r}
				  & 1 & 2 \\
				1 & ? & ? \\
				2 & ? & ? \\
			\end{tabular}		
		
		
		
		
		
		
		
		
		
		
		
		
		
		
		
		
	
		
		
		
	\begin{tiny}
	\copyright\ Joint-Troll-Expert-Group (JTEG) 2015
	\end{tiny}
\end{document}